\subsection{Pagine e relative funzioni e visibilità}
\subsubsection{Pagine non richiedenti login}
\paragraph{Welcome page - \textit{index.html}}
La pagina di accoglienza sarà una pagina con pochissime funzionalità, destinata prevalentemente a fare da \say{vetrina} del servizio offerto. Da essa si potranno trovare i link a tutte le altre funzionalità.

Vi saranno due pulsanti: \verb|login| e \verb|register|, e la navbar per navigare invece la parte pubblica.

\paragraph{Registrazione - \textit{register.html}}
La pagina di registrazione sarà costituita da un form. Se si è loggati, si verrà reindirizzati automaticamente alla pagina di login. Altrimenti, ci si potrà registrare. Se la registrazione va a buon fine si viene reindirizzati alla pagina di login, altrimenti viene mostrato il messaggio d'errore restituito dal backend.

\paragraph{Login - \textit{login.html}}
La pagina di login sarà costituita da un form. Se si è già loggati si viene reindirizzati alla pagina richiesta mediante un parametro, altrimenti ci si può loggare. In ogni caso saranno emessi appositi messaggi per segnalare lo stato.

\paragraph{Ricerca - \textit{search.html}}
La pagina di ricerca consentirà di effettuare una ricerca per vari campi. Una volta cercato, verranno mostrati i primi risultati di ogni categoria. Se si desidera ottenere maggiori risultati di quella categoria, i parametri vengono ristretti mentre avviene la redirezione a una pagina apposita.

\paragraph{Not Found - \textit{not\_found.html}}
Questa pagina sarà quella mostrata ogniqualvolta sia stata richiesta una pagina non esistente.

\subsubsection{Pagine richiedenti login}
\paragraph{Playlist - \textit{playlists.html}}
In questa pagina si potranno cercare le playlist pubbliche o comunque condivise con sé, e crearne di nuove.

\paragraph{Groups - \textit{groups.html}}
Questa pagina consentirà la creazione, la visualizzazione e il filtraggio dei gruppi.

\paragraph{Spiegazione di una playlist o di un gruppo \textit{explainPlaylist.html|explainGroup.html}}
Come la pagina di \verb|describe|, ma per playlist o gruppi. Dovranno includere l'unirsi e l'uscire dai gruppi, l'unire o il rimuovere una playlist a un gruppo e una canzone ad una playlist.
\paragraph{Profilo - \textit{profile.html}}
Questa pagina includerà tutte le informazioni legate all'utente, come preferiti, dati personali, playlist seguite o possedute, gruppi in cui si è o posseduti.


\subsection{Navbar condivisa tra tutte le pagine}
Tutte le pagine avranno accesso a una navbar condivisa, costituita dai seguenti elementi.
\subsubsection{Pulsante \textit{SNM}}
Permette di tornare alla pagina \verb|index.html|.
\subsubsection{Pulsante \textit{Playlists}}
Consente di tornare alla pagina delle playlist pubbliche.
\subsubsection{Pulsante \textit{Search}}
Consente di tornare alla pagina di ricerca.
\subsubsection{Pulsante \textit{Groups}}
Consente di tornare alla pagina dove sono mostrati i gruppi, ed eventualmente visualizzare le informazioni su di essi.
\subsubsection{Dropdown \textit{User}}
Esso conterrà diverse operazioni sul profilo, tra cui le seguenti:
\paragraph{Register}Visibile solo a chi non è loggato, permette di registrarsi.
\paragraph{Login}Visibile solo a chi non è loggato, permette di loggarsi.
\paragraph{Logout}Visibile solo a chi è loggato, permette di effettuare il logout.
\paragraph{My Favorites}Visibile solo a chi è loggato, permette di andare al profilo, nella sezione dedicata ai preferiti.
\paragraph{My Playlists}Visibile solo a chi è loggato, permette di andare al profilo, nella sezione dedicata alle playlists.
\paragraph{My Groups}Visibile solo a chi è loggato, permette di andare al profilo, nella sezione dedicata ai gruppi.
\paragraph{Profile}Visibile solo a chi è loggato, permette di andare al profilo.
\newpage