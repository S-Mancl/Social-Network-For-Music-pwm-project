\subsection{Pagine e relative funzioni e visibilità}
\subsubsection{Welcome page - \textit{index.html}}
La pagina di accoglienza sarà una pagina con pochissime funzionalità, destinata prevalentemente a fare da \say{vetrina} del servizio offerto.

Vi saranno due pulsanti: \verb|login| e \verb|register|, e la navbar per navigare invece la parte pubblica. Se si è già loggati, nel load si verrà reindirizzati alla pagina dove sono mostrate le playlist pubbliche o a quella dove è possibile effettuare ricerche.

\subsubsection{Registrazione - \textit{register.html}}
La pagina di registrazione sarà costituita da un form. Se si è loggati, si verrà reindirizzati automaticamente alla pagina di login. Altrimenti, ci si potrà registrare. Se la registrazione va a buon fine si viene reindirizzati alla pagina di login, altrimenti viene mostrato il messaggio d'errore restituito dal backend.

\subsubsection{Login - \textit{login.html}}
La pagina di login sarà costituita da un form. Se si è già loggati si viene reindirizzati alla pagina di visualizzazione dati pubblici, altrimenti ci si può loggare. In ogni caso saranno emessi appositi messaggi per segnalare lo stato.

\subsubsection{Ricerca - \textit{search.html}}
La pagina di ricerca consentirà di effettuare una ricerca per vari campi. Il meccanismo è ancora da studiare nei dettagli.

\subsubsection{Playlist - \textit{playlists.html}}
In questa pagina si potranno vedere le playlist pubbliche.

\subsubsection{MyPlaylists - \textit{myplaylists.html}}
In questa pagina si potranno vedere le proprie playlist, divise tra private, pubbliche e quelle altrui che vengono seguite.

\subsection{Navbar condivisa tra tutte le pagine}
Tutte le pagine avranno accesso a una navbar condivisa, costituita dai seguenti elementi.
\subsubsection{Pulsante \textit{SNM}}
Consente di tornare alla pagina delle playlist pubbliche.
\subsubsection{Pulsante \textit{Search}}
Consente di tornare alla pagina di ricerca.
\subsubsection{Pulsante \textit{Groups}}
Consente di tornare alla pagina dove sono mostrati i gruppi, ed eventualmente visualizzare le informazioni su di essi.
\subsubsection{Dropdown \textit{User}}
Esso conterrà diverse operazioni sul profilo, tra cui le seguenti:
\begin{center}
    \begin{tabular}[h]{ c || c | c }
        \textbf{Nome}&\textbf{Cosa fa}&\textbf{Condizioni di visualizzazione}\\
        \hline
        \textit{login}&Redirige alla pagina di login&Non essere loggati\\
        \textit{register}&Redirige alla pagina di registrazione&Sempre visibile\\
        \textit{profile}&Redirige alla pagina del profilo&Essere loggati\\
        \textit{logout}&Effettua il logout e redirige alla \say{vetrina}&Essere loggati\\
        \textit{my playlists}&Redirige alla pagina privata delle playlist personali&Essere loggati\\
    \end{tabular}
\end{center}
\newpage