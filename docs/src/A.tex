Di seguito saranno spiegate alcune scelte implementative relative a varie sezioni, iniziando da quelle riguardanti la struttura dati nel database.
\invisiblesubsection{Struttura dati nel database}
Il database prevede tre collezioni, ciascuna costituita da documenti con una data struttura:

\begin{figure}[h]
    \begin{minipage}{.30\textwidth}
        \textbf{Users}:

        \begin{lstlisting}[language=JavaScript]
{
    "_id": {...},
    "name": "First Name",
    "surname": "Surname",
    "userName": "userName", UNIQUE
    "email": "a@b.c", UNIQUE
    "birthDate": "2003-10-01",
    "favoriteGenres": [],
    "password": "...",
    "favorites": {
        "album": [],
        "artist": [],
        "audiobook": [],
        "episode": [],
        "show": [],
        "track": []
    },
    "playlistsFollowed": [],
    "playlistsOwned": [],
    "groupsFollowed": [],
    "groupsOwned": []
}
        \end{lstlisting}
    \end{minipage}
    \hfill
    \begin{minipage}{.30\textwidth}
        \textbf{Playlists}:

        \begin{lstlisting}[language=JavaScript]
{
    "_id":{...},
    "name": "myList", UNIQUE
    "description": "this is a playlist about old finnish songs",
    "tags":[
        "finnish",
        "old",
        "42"
    ],
    "visibility": true,
    "owner": "userName"
}








.
        \end{lstlisting}
    \end{minipage}
    \hfill
    \begin{minipage}{.30\textwidth}
        \textbf{Groups}:

        \begin{lstlisting}[language=JavaScript]
{
    "_id":{...},
    "name":"myGroup", UNIQUE
    "decription":"This is a group for lovers of classical music. Join this group to gain access to more than 15 playlists!",
    "playlistsShared":[
        "classicalMusic",
        "musicaClassica",
        ...
    ],
    "owner":"userName",
    "users":[
        "userName",
        "user42",
        "lambda",
        "SophosIoun",
        ...
    ]
}

        \end{lstlisting}
    \end{minipage}
\end{figure}
\begin{figure}[h]
    \begin{minipage}{.30\textwidth}
        Dove \verb|playlistsFollowed|, \verb|playlistsOwned|, \verb|groupsFollowed|, \verb|groupsOwned| referenziano gruppi o playlist nelle altre collezioni.
    \end{minipage}
    \hfill
    \begin{minipage}{.30\textwidth}
        Dove \verb|owner| referenzia \verb|userName|, che è lo \verb|userName| di uno \verb|user|. Si noti che una playlist non sa da quali utenti è seguita o in quali gruppi sia inserita.
    \end{minipage}
    \hfill
    \begin{minipage}{.30\textwidth}
        Dove \verb|owner| e \verb|users| referenziano \verb|userName| nella collezione \verb|users|, mentre \verb|playlistsShared| è un array di referenze ai nomi delle \verb|playlists| condivise con quel gruppo.
    \end{minipage}
\end{figure}
In tutti e tre i casi i campi indicati come \verb|UNIQUE| permettono di fare riferimento da altri. Laddove l'\verb|email| è solo usata per la login, tutti e tre i campi \verb|name| (\verb|user.userName|, \verb|playlist.name| e \verb|group.name|) hanno lo scopo di essere i riferimenti per documenti nelle altre collezioni.

\idea{So bene che alcune informazioni sono duplicate (in quanto sarebbe possibile avere le informazioni complete anche senza riportare sia gli users nei gruppi che i gruppi negli users, ad esempio) tuttavia queste informazioni sono presenti in entrambi i casi per una questione di facilità di uso delle informazioni e per ridurre il numero di chiamate al backend o al database necessarie per svolgere ogni operazione, al modico prezzo di un po' di spazio aggiuntivo occupato.}
Si noti poi che \verb|playlists.songs| è un array di oggetti costituiti come segue:
\begin{lstlisting}[language=JavaScript]
{
    "titolo":"15 secondi di muratori al lavoro",
    "durata":15000,
    "cantante":"Me medesimo",
    "anno_di_pubblicazione":"2023",
}
\end{lstlisting}
Mentre ogni preferito è salvato come un oggetto contenente titolo e id, tutto questo nell'apposito array.
\invisiblesubsection{Backend}
\invisiblesubsubsection{Accesso alle informazioni riservate}
Alcune informazioni non dovevano diventare pubbliche, ossia venire direttamente a contatto (per esempio venendovi inserite) con il frontend.

A questo scopo ho optato per l'utilizzo di un file \verb|.env|, prontamente inserito nel \verb|.gitignore| e che sarà consegnato a fianco della repo, contenente queste informazioni secondo il seguente formato:
\begin{lstlisting}[language=JavaScript]
MONGONAME=XXXXXXXXXXXX
MONGOPASSWORD=XXXXXXXXXXXX
PORT=3000
CLIENT_ID=XXXXXXXXXXXX
CLIENT_SECRET=XXXXXXXXXXXX
SECRET=XXXXXXXXXXXX
\end{lstlisting}
Dove \verb|SECRET| è la chiave per firmare i \verb|JWT|, \verb|MONGONAME| e \verb|MONGOPASSWORD| sono relativi all'utilizzo di MongoDB, \verb|CLIENT_ID| e \verb|CLIENT_SECRET| servono per Spotify e \verb|PORT| può essere usato per cambiare il numero della porta su cui viene esposto il servizio.

Per accedervi viene usato il package \verb|DotEnv|, di cui parlo più approfonditamente nella sezione \verb|tecnologie|.
\invisiblesubsubsection{JWT nei cookies anziché dati dell'utente nel local storage}
\idea{L'utilizzo dei JWT nei cookies permette di eseguire più facilmente e in modo più sicuro i controlli se l'utente è loggato o meno}
Voglio realizzare una autenticazione lievemente più sicura di quella richiesta. Per ottenere questo scopo utilizzo i JWT (con un apposito package di cui parlerà nella sezione \verb|tecnologie|), firmati con una chiave presente nel file \verb|.env|, e li salvo nei cookie in modo che vengano passati in automatico ad ogni richiesta.

Si avrà quindi il logout nella forma di un \verb|res.clearCookie('token')| e un controllo sull'identità dell'utente mediante un codice come quello che segue:
\begin{lstlisting}[language=JavaScript]
async function nomeMetodo(req,res){
    var pwmClient = await new mongoClient(mongoUrl).connect()
    const token = req.cookies.token
    if(token == undefined) res.status(401).json({"reason": `Invalid login`})
    else{
        jwt.verify(token,process.env.SECRET, async (err,decoded) =>{
            if(err){
                res.status(401).json(err)
                pwmClient.close()
            }
            else{
                //vero e proprio codice del metodo
                pwmClient.close()
            }
        })
    }
}
\end{lstlisting}
\alert{Questo sistema è sempre vulnerabile ad un possibile \say{furto} dei cookie: qualora questi venissero rubati è possibile per un utente inserirsi al posto di un altro, cambiare email e password e rendere impossibile all'utente originario l'accesso. Questo è un problema che non ho risolto in quanto ho ritenuto fuori dalle finalità di questo progetto, tuttavia sono consapevole del problema}
\invisiblesubsubsection{Token Object e perform function}
La richiesta (per quasi ogni informazione) di un token valido che scade ogni ora per operare con Spotify mi ha fatto optare per la creazione di un oggetto \verb|Token|, come nel seguente brano di codice:
\begin{lstlisting}[language=JavaScript]
var token = {
    value: "none",
    expiration: 42,
    regenAndThen : function(func_to_apply,paramA,paramB){
        fetch(baseUrls.token, {
            method: "POST",
            headers: {
            Authorization: "Basic " + btoa(`${process.env.CLIENT_ID}:${process.env.CLIENT_SECRET}`),
            "Content-Type": "application/x-www-form-urlencoded",
            },
            body: new URLSearchParams({ grant_type: "client_credentials" }),
        })
        .then((response) => response.json())
        .then((tokenResponse) => {
            this.expiration = new Date().getTime(); //ms
            this.value=tokenResponse.access_token
            func_to_apply(paramA,paramB)
        })                   
    },
    hasExpired : function(){
        if(((new Date().getTime()-this.expiration)/1000/60)>=59){
            return true;
        }
        else return false;
    }
}
\end{lstlisting}
Questo oggetto presenta una serie di caratteristiche: esso ha due attributi, \verb|value| che rappresenta il valore corrente del token, all'inizio un valore fasullo, e \verb|expiration| che permette di sapere il momento in cui è stato ottenuto, all'inizio un valore spurio (\verb|42|) così da essere sicuri che al primo uso esso venga ricaricato.

Esso ha inoltre due metodi.
Abbiamo infatti \verb|hasExpired| permette di sapere se sono passati più di (o esattamente) 59 minuti dalla scadenza.

\idea{Ho impostato come tempo limite 59 minuti e non un'ora perché le operazioni richiedono tempo, pertanto un'ora non sarebbe stato utile in quanto avrebbe potuto scadere prima del completamento della richiesta. In questo modo, invece, dovrebbe essere sempre possibile avere dei token validi.}

Vi è poi \verb|regenAndThen| che consiste in un applicatore: esso riceve tre parametri (\verb|func_to_apply|, funzione; \verb|paramA| e \verb|paramB| parametri da passare a quella funzione) ed effettua le seguenti operazioni:

\begin{enumerate}
    \item Rigenera il token
    \item Sostituisce il valore del token e il momento in cui è stato generato a quelli precedenti
    \item Chiama la funzione con i parametri passati
\end{enumerate}
\paragraph{perform} Il token object risulta di gran lunga più pratico nel momento in cui è utilizzato insieme alla funzione \verb|perform|, definita come segue:
\begin{lstlisting}[language=JavaScript]
function perform(questo,paramA,paramB){
    if(token.hasExpired()) token.regenAndThen(questo,paramA,paramB)
    else questo(paramA,paramB)
}
\end{lstlisting}
Questa funzione è ancora una volta un applicatore, in particolare nei confronti del parametro-funzione \verb|questo|. Esso verifica se il token è scaduto sfruttando gli appositi metodi dell'oggetto \verb|token|. In caso sia scaduto chiama l'applicatore \verb|regenAndThen|, altrimenti esegue direttamente, così da non rigenerare il token ad ogni chiamata, ma solo quando necessario.
\invisiblesubsubsection{Eliminazione di documenti e integrità delle referenze}
\idea{Non voglio dovermi ritrovare con dati inconsistenti o con riferimenti pendenti all'interno del database, perciò preferisco realizzare del codice molto inefficiente piuttosto che poi dover interpretare dati incompleti, parziali o superflui}
Ci sono tre richieste di tipo \verb|delete| che si possono fare all'API. Queste richieste, visibili anche nello swagger di cui all'Appendice B, sono \verb|app.delete('/playlist/:name')|, \verb|app.delete('/group/:name')| e \verb|app.delete('/user')|.

Quando avviene una chiamata a questi endpoint sarebbe facile risolverla con un banale \verb|deleteOne|. Questo, però, genererebbe problemi: dato che i documenti si referenziano tra loro creare dei riferimenti pending sarebbe impossibile.

Per eliminare quelli ho optato per una strategia divisa in più fasi, cercando sempre di assicurarmi che lo svolgimento erroneo di una fase non potesse compromettere quelle successive, ma piuttosto le fermasse. Questo perché è meglio, a mio giudizio, eliminare dei riferimenti senza riuscire a eliminare l'oggetto in questione piuttosto che eliminare l'oggetto lasciando dei riferimenti pending.

Le soluzioni sono le seguenti: iniziamo dagli utenti.
\begin{lstlisting}[language=JavaScript]
async function deleteUser(req,res){ 
    ...
    let user = await pwmClient.db("pwm_project").collection('users').findOne({"email": decoded.email})
    //-3 elimino l'utente da ogni gruppo in cui e', ed elimino ogni gruppo che sia owned, rimuovendolo prima da ogni utente che sia in quel gruppo
    let allGroups = await pwmClient.db("pwm_project").collection("groups").find({}).toArray()
    for(let index in allGroups){
        let group = allGroups[index]
        if(group.users.some(element => element == user.userName)){
            group.users.splice(group.users.indexOf(user.userName),1)
            await pwmClient.db("pwm_project").collection('groups')
                .updateOne({"name":group.name},
                {$set:{"users":group.users}})
        }
    }
    //-3 bis elimino ogni gruppo posseduto dall'utente
    await pwmClient.db("pwm_project").collection("groups").deleteMany({"owner":user.userName})
    //-2 elimino ogni playlist dell'utente da ogni lista di playlist seguite altrui
    allGroups = await pwmClient.db("pwm_project").collection("groups").find({}).toArray()
    let allPlaylists = pwmClient.db("pwm_project").collection("playlists").find({"owner":user.userName}).toArray()
    let allUsers = pwmClient.db("pwm_project").collection("playlists")
        .find({"email":{$ne:decoded.email}}).toArray()
    for(let index1 in allPlaylists){
        let playlist = allPlaylists[index1]
        for(let index2 in allUsers){
            let utente = allUsers[index2]
            if(utente.playlistsFollowed.some(element => element == playlist.name)){
                utente.playlistsFollowed.splice(utente.playlistsFollowed.indexOf(playlist.name),1)
                await pwmClient.db("pwm_project").collection("users")
                    .updateOne({"email":utente.email},
                    {$set:{"playlistsFollowed":utente.playlistsFollowed}})
            }
        }
        for(let index in allGroups){
            let group = allGroups[index]
            if(group.playlistsShared.some(element => element == playlist.name)){
                group.playlistsShared.splice(group.playlistsShared.indexOf(playlist.name),1)
                await pwmClient.db("pwm_project").collection("groups")
                    .updateOne({"name":group.name},
                    {$set:{"playlistsShared":group.playlistsShared}})
            }
        }
    }
    //-2 bis elimino ogni playlist owned dall'utente
    await pwmClient.db("pwm_project").collection("playlists").deleteMany({"owner":utente.userName})
    //-1 elimino l'account dell'utente
    await pwmClient.db('pwm_project').collection('users').deleteOne({"email":decoded.email})
    //0. forse e' andato tutto liscio, e spero di non aver lasciato riferimenti pending da qualche parte
    res.status(200).json({"reason":"ok"}) 
    ...
}
\end{lstlisting}
Mentre per le playlist:
\begin{lstlisting}[language=Javascript]
async function deletePlaylist(req,res){
    ...
    let a = await pwmClient.db("pwm_project").collection('playlists').findOne({"name": validator.escape(req.params.name)})
    let userToUpdate = await pwmClient.db("pwm_project").collection('users').findOne({"email":decoded.email});
    if(a != null && a != undefined && isOwner(a,userToUpdate.userName)){
        //-3. per ogni utente, diverso dall'owner, elimino la playlist da quelle seguite, laddove presente.
        let allUsers = pwmClient.db("pwm_project").collection("users").find({"email":{$ne: decoded.email}}).toArray()
        try{
            for(let index in allUsers){
                let user = allUsers[index]
                if(user.playlistsFollowed.some(element => element == a.name)){
                    user.playlistsFollowed.splice(user.playlistsFollowed.indexOf(a.name),1)
                    await pwmClient.db("pwm_project").collection('users')
                        .updateOne({"email":user.email},
                        {$set:{"playlistsFollowed":user.playlistsFollowed}})
                }
            }
        }catch(e){}
        //-2.5. per ogni gruppo elimino la playlist da quelle seguite, laddove presente
        let allGroups = await pwmClient.db("pwm_project").collection("groups").find({}).toArray()
        try{
            for(let index in allGroups){
                let group = allGroups[index]
                if(group.playlistsShared.some(element => element == a.name)){
                    group.playlistsShared.splice(group.playlistsShared.indexOf(a.name),1)
                    await pwmClient.db("pwm_project").collection('groups')
                        .updateOne({"name":group.name},
                        {$set:{"playlistsShared":group.playlistsShared}})
                }
            }
        }catch(e){}
        //-2. elimino, dall'utente che la possedeva, la playlist, sia da quelle seguite che da quelle totali (aka assicuro integrita' referenziale)
        userToUpdate.playlistsOwned.splice(await userToUpdate.playlistsOwned.indexOf(a.name),1)
        userToUpdate.playlistsFollowed.splice(await userToUpdate.playlistsFollowed.indexOf(a.name),1)
        await pwmClient.db("pwm_project").collection('users')
            .updateOne({"email":decoded.email},
            {$set:{"playlistsFollowed":userToUpdate.playlistsFollowed,
            "playlistsOwned":userToUpdate.playlistsOwned}});
        //-1. elimino la playlist
        await pwmClient.db("pwm_project").collection('playlists').deleteOne(a)
        //se non e' esploso niente allora e' tutto okay, spero
        res.status(200).json({"reason":"done correctly"})
    }
    else res.status(400).json({"reason":"Probably you haven't specified the right params"})
    ...
}
\end{lstlisting}
E per i gruppi:
\begin{lstlisting}[language=JavaScript]
    async function deleteGroup(req,res){
    ...
    //elimino un gruppo
    let a = await pwmClient.db("pwm_project").collection('groups').findOne({"name": validator.escape(req.params.name)})
    //-3. elimino dagli utenti non owner ogni riferimento a quel gruppo
    let allUsers = await pwmClient.db("pwm_project").collection("users").find({"email":{$ne: decoded.email}}).toArray()
    for(let index in allUsers){
        let user = allUsers[index]
        if(user.groupsFollowed.some(element => element == a.name)){
            user.groupsFollowed.splice(user.groupsFollowed.indexOf(a.name),1)
            await pwmClient.db("pwm_project").collection('users')
                .updateOne({"email":user.email},
                {$set:{"groupsFollowed":user.groupsFollowed}})
        }
    }
    //-2. elimino dall'owner il gruppo, sia owned che followed
    let userToUpdate = await pwmClient.db("pwm_project").collection("users").findOne({"email":decoded.email})
    userToUpdate.groupsFollowed.splice(await userToUpdate.groupsFollowed.indexOf(a.name),1)
    userToUpdate.groupsOwned.splice(await userToUpdate.groupsOwned.indexOf(a.name),1)
    await pwmClient.db("pwm_project").collection('users')
        .updateOne({"email":decoded.email},
        {$set:{"groupsFollowed":userToUpdate.groupsFollowed,"groupsOwned":userToUpdate.groupsOwned}});
    //-1 elimino il gruppo
    await pwmClient.db("pwm_project").collection('groups').deleteOne(a)
    //0. forse e' andato tutto liscio
    res.status(200).json({"reason":"ok"})
    ...
}
\end{lstlisting}
\alert{Un problema simile si avrebbe anche in caso qualcuno desiderasse modificare lo username. Per risolverlo, ho reso impossibile modificare lo username, consentendo invece di modificare le credenziali usate per il login ossia solamente email e password}
\invisiblesubsection{Frontend}
Le scelte implementative effettuate riguardanti il frontend sono molteplici e molto ampie, pertanto verranno trattate separatamente. In questa sede saranno riportate solo le scelte che ritengo significative, per evitare di riportare nozioni a mio giudizio inutili.
\invisiblesubsubsection{Redirect e rischio di loop}
In svariate situazioni un redirect avrebbe potuto essere una ottima idea. Quando ad esempio qualcuno, senza essere loggato (magari usando un vecchio link o avendo salvato nei \verb|bookmarks| l'indirizzo) cerca di accedere al proprio profilo, egli viene reindirizzato alla login, e successivamente al proprio profilo.

Una scelta simile, pensavo inizialmente, si potrebbe applicare anche nel caso di playlist per le quali non ci fosse la possibilità, causa permessi mancanti, di visualizzarle. Alla fine ho optato invece per non farlo perché il rischio era che un utente, loggato, non avesse accesso alla playlist in quanto privata, venisse quindi reindirizzato alla login. Constatando che è già loggato, la pagina di login reindirizzerebbe verso la pagina richiesta inizialmente, pur non essendo di fatto cambiate le credenziali, e pertanto la pagina delle playlist avrebbe nuovamente rediretto alla login, generando un ciclo.
\alert{Si noti che anche nel caso peggiore il ciclo non avrebbe potuto durare più di un'ora causa scadenza dei token JWT. Il tempo sarebbe comunque stato decisamente eccessivo.}
\invisiblesubsubsection{Eventi}
In tre diverse componenti del Javascript frontend ho avuto necessità di creare degli eventi.
\paragraph{Responsiveness} La pagina che mostra le informazioni ottenute da Spotify deve essere responsive. Questo normalmente, usando le classi di Bootstrap, si risolverebbe senza bisogno di particolari menzioni. In questo caso, tuttavia, il mio appoggiarmi a \href{https://flagcdn.com/}{https://flagcdn.com/} rende necessaria una operazione diversa: le flag vengono richieste con una dimensione fissa, pertanto all'aumentare dello schermo oltre i 2k rischiano di diventare microscopiche.

Ho pertanto optato per la creazione dell'evento sotto, in grado di recuperare le informazioni sullo stato di cui occorre recuperare la bandiera (mediante una apposita regular expression) e aumentare o ridurre la dimensione in base alle necessità.
\begin{lstlisting}[language=JavaScript]
window.addEventListener('resize', () =>{
    const re = /\/.....\//
    let a = document.getElementsByClassName('flag')
    for (let i=0;i<a.length;i++){
        let element = a[i].src
        let code = element.split(re)[1].split(".png")[0]
        a[i].src = `https://flagcdn.com/${window.screen.availWidth<2000?"16x12":"64x48"}/${code}.png`
    }
});
\end{lstlisting}
\paragraph{Selezione dei gruppi} Mediante la proprietà \verb|oninput| di un \verb|input| di tipo \verb|text| sono in grado di filtrare dinamicamente i gruppi in modo che i titoli contengano come sottostringa il testo inserito. Si può notare questa cosa in azione nella pagina \verb|groups.html|.
\begin{lstlisting}[language=JavaScript]
function changing(){
    let query = document.getElementById('search').value.toLowerCase()
    for(let i =0;i<groupList.length;i++){
        let card = document.getElementById('card-groups-followed-'+i)
        if(!card.getElementsByClassName(`card-title`)[0].innerHTML.toLowerCase().includes(query)) card.classList.add('d-none')
        else card.classList.remove('d-none')

    }
}
\end{lstlisting}
\paragraph{Cambio della voce selezionata in un select} Per verificare che una canzone appartenga o meno ad una playlist, e quindi se essa possa o meno essere aggiunta o rimossa, ho creato un evento, come segue:
\begin{lstlisting}[language=JavaScript]
document.querySelector('#floatingSelect').addEventListener('change', () =>{
    let playlist = document.querySelector('#floatingSelect').value
    let id = params.get('value')
    fetch(`/playlist/info/${playlist}`).then(async a => {
        if(a.ok){
            response = await a.json()
            if(!response.songs.some(element => element.id == id)){
                //button to remove it
                console.log(document.getElementById('the-mystic-button').onclick)
                document.getElementById('the-mystic-button').innerHTML='Add it!'
                document.getElementById('the-mystic-button').classList.add('text-bg-success')
            }
            else{
                //button to add it
                console.log(document.getElementById('the-mystic-button').onclick)
                document.getElementById('the-mystic-button').innerHTML='Remove it!'
                document.getElementById('the-mystic-button').classList.add('text-bg-danger')
            }
        }
    })
});
\end{lstlisting}
\newpage